\documentclass[10pt]{article}
% generated by Madoko, version 1.0.0-rc1
%mdk-data-line={1}


\usepackage[heading-base={2},section-num={False},bib-label={True}]{madoko2}
\usepackage{ctex}


\begin{document}



%mdk-data-line={15}
\mdxtitleblockstart{}
%mdk-data-line={15}
\mdxtitle{\mdline{15}基于用户评论的电影关键词提取工具}%mdk
\mdxauthorstart{}
%mdk-data-line={20}
\mdxauthorname{\mdline{20}丁戍}%mdk

%mdk-data-line={23}
\mdxauthoraddress{\mdline{23}13307130299}%mdk

%mdk-data-line={26}
\mdxauthoremail{\mdline{26}sding13@fudan.edu.cn}%mdk
\mdxauthorend\mdtitleauthorrunning{}{}\mdxtitleblockend%mdk

%mdk-data-line={17}
\begin{abstract}%mdk

%mdk-data-line={18}
\noindent\mdline{18}本工具以豆瓣电影为代表,未经监督训练,抓取用户评论,再通过用户的评论文章,使用 TF-IDF 来提取一部电影的关键词信息,并将关键字评分排序,展现为可视化数据图表。从数据抓取到数据展现,全程均实现了自动化。%mdk
%mdk
\end{abstract}%mdk

%mdk-data-line={21}
\section{\mdline{21}1.\hspace*{0.5em}\mdline{21}介绍}\label{sec-intro}%mdk%mdk

%mdk-data-line={23}
\noindent\mdline{23}目前,所有主流的电影资讯、评论、分享网站,为一部电影提供的类别区分仅有数十种左右,且相当模糊。电影作为数百分钟的音乐、画面、剧情故事载体,题材与形式的组合成千上万,大部分情况下并不能被粗暴地归为几类。并且,这样的区分方式十分不适合用户来整理、挑选电影。%mdk

%mdk-data-line={25}
\mdline{25}以世界上最大的电影资料库 IMDB 为例,其中用户评分最高\mdline{25}\mdfootnote{1}{%mdk-data-line={27}
%mdk-data-line={27}
\noindent\mdline{27}\href{http://www.imdb.com/chart/top?ref_=tt_awd}{Top Rated Movies - IMDB}\mdline{27}%mdk
\label{fn-imdbtop}%mdk%mdk
}\mdline{25}的电影《肖申克的救赎》(The Shawshank Redemption),只有 “犯罪(crime)” 和 “剧情(drama)” 两个类别\mdline{25}\mdfootnote{2}{%mdk-data-line={28}
%mdk-data-line={28}
\noindent\mdline{28}\href{http://www.imdb.com/title/tt0111161/?pf_rd_m=A2FGELUUNOQJNL\%26pf_rd_p=2239792642\%26pf_rd_r=18FXCTHFQCHA1WZXPT6Y\%26pf_rd_s=center-1\%26pf_rd_t=15506\%26pf_rd_i=top\%26ref_=chttp_tt_1\%23overview-top}{The Shawshank Redemption - IMDB}%mdk
\label{fn-imdbgen}%mdk%mdk
}\mdline{25}。而在最大的华语电影资料库豆瓣电影上,《肖申克的救赎》同样也只有这两个类别\mdline{25}\mdfootnote{3}{%mdk-data-line={29}
%mdk-data-line={29}
\noindent\mdline{29}\href{http://movie.douban.com/subject/1292052/}{肖申克的救赎 - 豆瓣}%mdk
\label{fn-doubangen}%mdk%mdk
}\mdline{25}。而且豆瓣上,官方提供的标准化主题关键词数据则有更加限。而用户自行添加的标签多为 “经典”、“励志”,有时又为电影年代如 “1949”,有时又为演员昵称,都未经过标准整理。%mdk

%mdk-data-line={31}
\mdline{31}本工具以豆瓣电影为代表,未经监督训练,通过用户的评论文章来提取一部电影的关键词信息,并将关键字评分排序,展现为数据图表。从数据抓取到数据展现的整个过程,均实现了自动化。例如同一部电影,本工具得到的结果如下:%mdk

%mdk-data-line={33}
\begin{figure}[h]%mdk
\begin{mdcenter}%mdk

%mdk-data-line={34}
\noindent\mdline{34}\includegraphics[keepaspectratio=true,width=26em]{xsk}{}\mdline{34}%mdk

%mdk-data-line={35}
\mdhr{}%mdk

%mdk-data-line={36}
\noindent\mdline{36}\mdcaption{\textbf{Figure~\mdcaptionlabel{1}.} \mdcaptiontext{《肖申克的救赎》自动提取关键词}}%mdk
%mdk
\end{mdcenter}\label{fig-xsk}%mdk
%mdk
\end{figure}%mdk

%mdk-data-line={39}
\noindent\mdline{39}此工具的主要特点有:%mdk

%mdk-data-line={41}
\begin{itemize}[noitemsep,topsep=\mdcompacttopsep]%mdk

%mdk-data-line={41}
\item\mdline{41}通过电影名,自动搜索并抓取所有评论信息,存入数据库;%mdk

%mdk-data-line={42}
\item\mdline{42}不同关键词的大小表示其权重,权重越大表示关键词越重要;%mdk

%mdk-data-line={43}
\item\mdline{43}提取剧情、题材相关的关键性名词、动词,例如上图的 “自由”、“监狱”、“希望”、“毅力”、“活着” 等。这些词构成了一部电影的骨干;%mdk

%mdk-data-line={44}
\item\mdline{44}提取形容电影节奏、画面、音乐、演员表演水平等等部分的形容词(上例中未出现);%mdk

%mdk-data-line={45}
\item\mdline{45}过滤了非电影内容本身相关的词语,例如 \mdline{45}\textquotedblleft{}出品年代\textquotedblright{}\mdline{45}、\mdline{45}\textquotedblleft{}影院\textquotedblright{}\mdline{45}、\mdline{45}\textquotedblleft{}IMAX\textquotedblright{}\mdline{45} 这些经常出现的标签;%mdk

%mdk-data-line={46}
\item\mdline{46}整理出出现关键词的句子,可交互查看关键词的详情。%mdk
%mdk
\end{itemize}%mdk

%mdk-data-line={48}
\section{\mdline{48}2.\hspace*{0.5em}\mdline{48}细节实现}\label{section}%mdk%mdk

%mdk-data-line={50}
\subsection{\mdline{50}2.1.\hspace*{0.5em}\mdline{50}搜索、抓取、规范化数据}\label{section}%mdk%mdk

%mdk-data-line={52}
\noindent\mdline{52}本项目独立实现了一个爬虫,并接入了豆瓣电影的各个 API。运行步骤如下:%mdk

%mdk-data-line={54}
\begin{enumerate}[noitemsep,topsep=\mdcompacttopsep]%mdk

%mdk-data-line={54}
\item\mdline{54}第一步,爬虫通过电影名称搜索到其唯一编号(search 接口);%mdk

%mdk-data-line={55}
\item\mdline{55}在豆瓣电影页面获取其基本信息(details 接口),如演员、类型、短评入口、长评入口;%mdk

%mdk-data-line={56}
\item\mdline{56}分别抓取一定数量的短评(comments 接口)、长评(reviews 接口);%mdk

%mdk-data-line={57}
\item\mdline{57}将各个评论的评分(一至五星)、内容、用户赞同数量等数据清理成可用的格式(去掉空格,转换成评分百分比等等)%mdk
%mdk
\end{enumerate}%mdk

%mdk-data-line={59}
\begin{figure}[tbp]%mdk
\begin{mdcenter}%mdk

%mdk-data-line={60}
\noindent\mdline{60}\includegraphics[keepaspectratio=true,width=\dimwidth{1.00}]{comments}{}\mdline{60}%mdk

%mdk-data-line={61}
\mdhr{}%mdk

%mdk-data-line={62}
\noindent\mdline{62}\mdcaption{\textbf{Figure~\mdcaptionlabel{2}.} \mdcaptiontext{短评}}%mdk
%mdk
\end{mdcenter}\label{fig-comments}%mdk
%mdk
\end{figure}%mdk

%mdk-data-line={66}
\begin{figure}[tbp]%mdk
\begin{mdcenter}%mdk

%mdk-data-line={67}
\noindent\mdline{67}\includegraphics[keepaspectratio=true,width=\dimwidth{1.00}]{reviews}{}\mdline{67}%mdk

%mdk-data-line={68}
\mdhr{}%mdk

%mdk-data-line={69}
\noindent\mdline{69}\mdcaption{\textbf{Figure~\mdcaptionlabel{3}.} \mdcaptiontext{长评}}%mdk
%mdk
\end{mdcenter}\label{fig-reviews}%mdk
%mdk
\end{figure}%mdk

%mdk-data-line={72}
\noindent\mdline{72}爬虫脚本使用 Node.js 实现,所有接口都支持并发、缓存等特性,代码位于 \mdline{72}\mdcode{core/api.js}\mdline{72}。%mdk

%mdk-data-line={74}
\subsection{\mdline{74}2.2.\hspace*{0.5em}\mdline{74}分析}\label{section}%mdk%mdk

%mdk-data-line={76}
\subsubsection{\mdline{76}2.2.1.\hspace*{0.5em}\mdline{76}短评}\label{section}%mdk%mdk

%mdk-data-line={78}
\noindent\mdline{78}根据粗略观察,短评限于其字数,没有办法围绕一个重点来展开讨论。以《肖申克的救赎》为例,排名第一的短评为%mdk

%mdk-data-line={80}
\begin{quote}%mdk

%mdk-data-line={80}
\noindent\mdline{80}忒经典的东西,我要带去我的坟墓%mdk
%mdk
\end{quote}%mdk

%mdk-data-line={82}
\noindent\mdline{82}其中关键词可能有 \mdline{82}\textquotedblleft{}经典\textquotedblright{}\mdline{82} 和 \mdline{82}\textquotedblleft{}坟墓\textquotedblright{}\mdline{82},而 TF-IDF 算法会将 \mdline{82}\textquotedblleft{}坟墓\textquotedblright{}\mdline{82} 视为第一关键词。而根据语义,\mdline{82}\textquotedblleft{}坟墓\textquotedblright{}\mdline{82} 和电影主题并不相关。%mdk

%mdk-data-line={84}
\mdline{84}大多数短评围绕的内容比较五花八门,因此其中名词并无太多利用价值。而大部分用户写一句话短评的时候,其中形容词几乎都用于形容电影,例如这个短评:%mdk

%mdk-data-line={86}
\begin{quote}%mdk

%mdk-data-line={86}
\noindent\mdline{86}“这是一部男人必看的电影。” 人人都这么说。但单纯从性别区分,就会让这电影变狭隘。《肖申克的救赎》突破了男人电影的局限,通篇几乎充满令人难以置信的温馨基调,而电影里最伟大的主题是 “希望”。 当我们无奈地遇到了如同肖申克一般囚禁了心灵自由的那种囹圄,我们是无奈的老布鲁克,灰心的瑞德,还是智慧的安迪?运用智慧,信任希望,并且勇敢面对恐惧心理,去打败它? 经典的电影之所以经典,因为他们都在做同一件事——让你从不同的角度来欣赏希望的美好。%mdk
%mdk
\end{quote}%mdk

%mdk-data-line={88}
\noindent\mdline{88}其中大多数形容词,例如 \mdline{88}\textquotedblleft{}温馨\textquotedblright{}\mdline{88}、\mdline{88}\textquotedblleft{}无奈\textquotedblright{}\mdline{88}、\mdline{88}\textquotedblleft{}灰心\textquotedblright{}\mdline{88}、\mdline{88}\textquotedblleft{}智慧\textquotedblright{}\mdline{88}、\mdline{88}\textquotedblleft{}勇敢\textquotedblright{}\mdline{88}、……都是电影中重要的主题。%mdk

%mdk-data-line={90}
\mdline{90}另一类短评,会围绕电影本身形式的一些方面进行描述,例如 \mdline{90}\textquotedblleft{}摄影构图精美\textquotedblright{}\mdline{90}、\mdline{90}\textquotedblleft{}配乐恢宏\textquotedblright{}\mdline{90}、\mdline{90}\textquotedblleft{}剧情波折\textquotedblright{}\mdline{90} 等等。%mdk

%mdk-data-line={92}
\subsubsection{\mdline{92}2.2.2.\hspace*{0.5em}\mdline{92}长评}\label{section}%mdk%mdk

%mdk-data-line={94}
\noindent\mdline{94}长评论因为没有了字数限制,可以通篇围绕一个或多个主题来展开详细评论。这样带来的好处是,其主题会在通篇文字中多次出现。例如这部电影下排名第一的评论\mdline{94}\mdfootnote{4}{%mdk-data-line={100}
%mdk-data-line={100}
\noindent\mdline{100}\href{http://movie.douban.com/review/1000369/}{十年·肖申克的救赎}%mdk
\label{fn-revlink}%mdk%mdk
}\mdline{94}《十年·肖申克的救赎》,作者围绕信念、自由、友谊三个主题来进行详细讨论。这类文章往往会引导其下评论的方向,也为这三个主题。因此整个页面中 \mdline{94}\textquotedblleft{}自由\textquotedblright{}\mdline{94} 一词出现近 40 次。%mdk

%mdk-data-line={96}
\mdline{96}于是我们首先使用 TF-IDF 算法提取出全文关键词。接着,根据每篇长评之后的用户赞同数(例如本篇是:有用 6386 没用 166),进行评分。赞同数越多文章,其中关键字整体权重越大。%mdk

%mdk-data-line={98}
\mdline{98}最后,我们合并所有长评的关键词与其评分,进行排序。总体公式如下:%mdk
\label{score}%mdk
\noindent\mdline{102}\mdmathtag{(1)}\mdline{102}
\noindent\[%mdk-data-line={103}
s(word) = \sum_{\text{所有长评论}}{\log{(word \text{在此篇中的 TF-IDF 评分})} * \text{此篇权重}}
\]%mdk

%mdk-data-line={106}
%mdk-data-line={107}
\noindent\mdline{107}\textbf{Note}.
\mdline{108}权重定义为赞同数占所有长评赞同数之和的比。%mdk%mdk

%mdk-data-line={110}
\subsection{\mdline{110}2.3.\hspace*{0.5em}\mdline{110}实现}\label{section}%mdk%mdk

%mdk-data-line={112}
\noindent\mdline{112}在具体处理时,我们首先抓取了大量电影相关术语(共约 400 个),避免被视作关键词。例如:%mdk
\begin{mdpre}%mdk
\noindent票房\\
镜头\\
奥斯卡\\
暗箱\\
借位\\
功夫片\\
电影人\\
院线\\
戏院\\
近景\\
特写\\
调色\\
...%mdk
\end{mdpre}\noindent\mdline{130}这些词语从各个电影论坛抓起,文件位于 \mdline{130}\mdcode{core/movie.dict.utf8}\mdline{130} 中。另外还有常用的 stopwords,位于 \mdline{130}\mdcode{core/stopwords.utf8}\mdline{130} 中。

%mdk-data-line={132}
\mdline{132}接着我们排除了一些标签,比如时间、地点、人物这类常常出现的、但与电影剧情等无关的词\mdline{132}~[\mdcite{ictpo}{3}]\mdline{132}。%mdk

%mdk-data-line={134}
\mdline{134}分词和 TF-IDF 使用了 jieba 中文分词库\mdline{134}[\mdcite{jieba}{2}]\mdline{134}。这部分代码主要位于 \mdline{134}\mdcode{core/main.js}\mdline{134} 中。%mdk

%mdk-data-line={136}
\mdline{136}最后将所有数据传至前端,以 d3.js 渲染出来。%mdk

%mdk-data-line={138}
\subsection{\mdline{138}2.4.\hspace*{0.5em}\mdline{138}效果展示}\label{section}%mdk%mdk

%mdk-data-line={140}
\begin{figure}[tbp]%mdk
\begin{mdcenter}%mdk

%mdk-data-line={141}
\noindent\mdline{141}\includegraphics[keepaspectratio=true,width=\dimwidth{1.00}]{agan}{}\mdline{141}%mdk

%mdk-data-line={142}
\mdhr{}%mdk

%mdk-data-line={143}
\noindent\mdline{143}\mdcaption{\textbf{Figure~\mdcaptionlabel{4}.} \mdcaptiontext{阿甘正传}}%mdk
%mdk
\end{mdcenter}\label{fig-agan}%mdk
%mdk
\end{figure}%mdk

%mdk-data-line={144}
\begin{figure}[tbp]%mdk
\begin{mdcenter}%mdk

%mdk-data-line={145}
\noindent\mdline{145}\includegraphics[keepaspectratio=true,width=26em]{buda}{}\mdline{145}%mdk

%mdk-data-line={146}
\mdhr{}%mdk

%mdk-data-line={147}
\noindent\mdline{147}\mdcaption{\textbf{Figure~\mdcaptionlabel{5}.} \mdcaptiontext{布达佩斯大饭店}}%mdk
%mdk
\end{mdcenter}\label{fig-buda}%mdk
%mdk
\end{figure}%mdk

%mdk-data-line={148}
\begin{figure}[tbp]%mdk
\begin{mdcenter}%mdk

%mdk-data-line={149}
\noindent\mdline{149}\includegraphics[keepaspectratio=true,width=26em]{star}{}\mdline{149}%mdk

%mdk-data-line={150}
\mdhr{}%mdk

%mdk-data-line={151}
\noindent\mdline{151}\mdcaption{\textbf{Figure~\mdcaptionlabel{6}.} \mdcaptiontext{星球大战}}%mdk
%mdk
\end{mdcenter}\label{fig-star}%mdk
%mdk
\end{figure}%mdk

%mdk-data-line={152}
\begin{figure}[tbp]%mdk
\begin{mdcenter}%mdk

%mdk-data-line={153}
\noindent\mdline{153}\includegraphics[keepaspectratio=true,width=26em]{interstellar}{}\mdline{153}%mdk

%mdk-data-line={154}
\mdhr{}%mdk

%mdk-data-line={155}
\noindent\mdline{155}\mdcaption{\textbf{Figure~\mdcaptionlabel{7}.} \mdcaptiontext{星际穿越}}%mdk
%mdk
\end{mdcenter}\label{fig-interstellar}%mdk
%mdk
\end{figure}%mdk

%mdk-data-line={165}
\section{\mdline{165}3.\hspace*{0.5em}\mdline{165}实验}\label{section}%mdk%mdk

%mdk-data-line={167}
\noindent\mdline{167}可以发现,根据网络用户自发的评论,我们确实可以总结出许多官方未提供的关键词信息。在数据量足够多的情况下,我们可以做许多有趣的分析。%mdk

%mdk-data-line={169}
\mdline{169}比如通过比对大量科幻片,我们就能发现科幻电影中 \mdline{169}\textquotedblleft{}飞船\textquotedblright{}\mdline{169} 出现的几率特别高。因此可以反过来,在只给出影评的情况下来判断一部电影的类型。%mdk

%mdk-data-line={171}
\mdline{171}我们首先采用上文的方法,提取出若干关键词。再分别采用朴素贝叶斯和线性回归分类器做试验,由于数据量不够多,结果并不如期望那么好。图\mdline{171}~\mdref{fig-mov}{\mdcaptionlabel{8}}\mdline{171} 是对《肖申克的救赎》的关键词统计,以及根据关键词和线性回归分类器所做的电影成分分析。%mdk

%mdk-data-line={173}
\begin{figure}[tbp]%mdk
\begin{mdcenter}%mdk

%mdk-data-line={174}
\noindent\mdline{174}\includegraphics[keepaspectratio=true,width=20em]{mov}{}\mdline{174}%mdk

%mdk-data-line={175}
\mdhr{}%mdk

%mdk-data-line={176}
\noindent\mdline{176}\mdcaption{\textbf{Figure~\mdcaptionlabel{8}.} \mdcaptiontext{《肖申克的救赎》电影类别判断}}%mdk
%mdk
\end{mdcenter}\label{fig-mov}%mdk
%mdk
\end{figure}%mdk

%mdk-data-line={180}
\noindent\mdline{180}分类器的代码位于 \mdline{180}\mdcode{core/basic.js}\mdline{180},用于训练的电影列表位于 \mdline{180}\mdcode{index.js}\mdline{180},分类器目前训练出的结果在 \mdline{180}\mdcode{basic/classifier.json}\mdline{180} 中。%mdk

%mdk-data-line={182}
\section{\mdline{182}4.\hspace*{0.5em}\mdline{182}展望和总结}\label{section}%mdk%mdk

%mdk-data-line={184}
\noindent\mdline{184}目前大部分人观看电影是根据档期(当前上映)。这种方式很难与自己喜爱的电影 \mdline{184}\textquotedblleft{}不期而遇\textquotedblright{}\mdline{184}。而如果数据量足够多,即使重口难调,我们也可以采用题材、元素的方式来挑选我们想看的电影。例如我们可以寻找 \mdline{184}\textquotedblleft{}有梦境、枪战等元素的科幻片\textquotedblright{}\mdline{184}(盗梦空间)。%mdk

%mdk-data-line={186}
\mdline{186}通过一学期自然信息处理的学习,我从零开始接触到了文本信息处理的很多细节,学到了分句分词、语义、标注、分割、分类器等等概念,并自己动手实现、使用了这样一套工具流程。%mdk

%mdk-data-line={188}
\mdline{188}本文想法和实现上参考了论文\mdline{188}[\mdcite{llpfc}{1}]\mdline{188}、\mdline{188}[\mdcite{rwup}{4}]\mdline{188}。最后感谢黄萱菁老师的辛勤付出!%mdk

%mdk-data-line={191;out/NLP-bib.bbl.mdk:1}
%mdk-data-line={191;out/NLP-bib.bbl.mdk:2}
\mdsetrefname{References}%mdk
{\mdbibindent{0}%mdk
\begin{thebibliography}{4}%mdk

%mdk-data-line={191;out/NLP-bib.bbl.mdk:5}
\bibitem{llpfc}\mdline{191;out/NLP-bib.bbl.mdk:6}Noah A.\mdline{191;out/NLP-bib.bbl.mdk:6}~\mdline{191;out/NLP-bib.bbl.mdk:6}Smith David\mdline{191;out/NLP-bib.bbl.mdk:6}~\mdline{191;out/NLP-bib.bbl.mdk:6}Bamman, Brendan\mdline{191;out/NLP-bib.bbl.mdk:6}~\mdline{191;out/NLP-bib.bbl.mdk:6}O’Connor.
\mdline{191;out/NLP-bib.bbl.mdk:7}\newblock \mdline{191;out/NLP-bib.bbl.mdk:7} Learning latent personas of film characters.
\mdline{191;out/NLP-bib.bbl.mdk:8}\newblock \mdline{191;out/NLP-bib.bbl.mdk:8} Master\mdline{191;out/NLP-bib.bbl.mdk:8}'\mdline{191;out/NLP-bib.bbl.mdk:8}s thesis, School of Computer Science, Carnegie Mellon
  University, Pittsburgh, PA 15213, USA, 2012.\label{llpfc}%mdk%mdk

%mdk-data-line={191;out/NLP-bib.bbl.mdk:12}
\bibitem{jieba}\mdline{191;out/NLP-bib.bbl.mdk:13}fxsjy.
\mdline{191;out/NLP-bib.bbl.mdk:14}\newblock \mdline{191;out/NLP-bib.bbl.mdk:14} 结巴中文分词.
\mdline{191;out/NLP-bib.bbl.mdk:15}\newblock \mdline{191;out/NLP-bib.bbl.mdk:15} URL \mdline{191;out/NLP-bib.bbl.mdk:15}\href{https://github.com/fxsjy/jieba}{https://github.com/fxsjy/jieba}\mdline{191;out/NLP-bib.bbl.mdk:15}.\label{jieba}%mdk%mdk

%mdk-data-line={191;out/NLP-bib.bbl.mdk:18}
\bibitem{ictpo}\mdline{191;out/NLP-bib.bbl.mdk:19}luw2007.
\mdline{191;out/NLP-bib.bbl.mdk:20}\newblock \mdline{191;out/NLP-bib.bbl.mdk:20} Ictpos3.0 词性标记集.
\mdline{191;out/NLP-bib.bbl.mdk:21}\newblock \mdline{191;out/NLP-bib.bbl.mdk:21} URL \mdline{191;out/NLP-bib.bbl.mdk:21}\href{https://gist.github.com/luw2007/6016931}{https://gist.github.com/luw2007/6016931}\mdline{191;out/NLP-bib.bbl.mdk:21}.\label{ictpo}%mdk%mdk

%mdk-data-line={191;out/NLP-bib.bbl.mdk:24}
\bibitem{rwup}\mdline{191;out/NLP-bib.bbl.mdk:25}Eduard\mdline{191;out/NLP-bib.bbl.mdk:25}~\mdline{191;out/NLP-bib.bbl.mdk:25}Hovy Michael\mdline{191;out/NLP-bib.bbl.mdk:25}~\mdline{191;out/NLP-bib.bbl.mdk:25}Fleischman.
\mdline{191;out/NLP-bib.bbl.mdk:26}\newblock \mdline{191;out/NLP-bib.bbl.mdk:26} Recommendations without user preferences: A natural language
  processing approach.
\mdline{191;out/NLP-bib.bbl.mdk:28}\newblock \mdline{191;out/NLP-bib.bbl.mdk:28} Master\mdline{191;out/NLP-bib.bbl.mdk:28}'\mdline{191;out/NLP-bib.bbl.mdk:28}s thesis, USC Information Science Institute, 4676 Admiralty
  Way, Marina del Rey, CA 90292-6695.\label{rwup}%mdk%mdk
\par%mdk
\end{thebibliography}}%mdk%mdk

%mdk-data-line={194}
\begin{mdbmargintb}{4em}{}%mdk
\begin{mdflushright}%mdk
{\tiny\mdline{195}Created with~\href{https://www.madoko.net}{Madoko.net}.}%mdk
\end{mdflushright}%mdk
\end{mdbmargintb}%mdk%mdk


\end{document}
